\documentclass{article}
\author{Mergim Maraj Martinez}
\title{Wellenmechanik}
\date{01. November 2014}
\begin{document}
\maketitle

\section{Ich war hier}
Hallo Mergim, ich war hier. Gruss Rebecca Wieland (Rebeggy)
\section{Die De-Broglie Beziehung}
Die De-Broglie-Beziehung
Die Quantentheorie besagt, dass Licht einerseits als Welle, andererseits auch als Strahlen von Teilchen – den Photonen. Dieser Wellen-Teilchen-Dualismus stellt zwei Theorien dar, die beide den gleichen Sach-verhalt besser oder schlechter darstellen (je nach Fall). Jedoch darf dies nicht falsch verstanden werden – Licht ist nicht einmal Teilchen, einmal Welle. Nur kennen wir keine besseren Modelle bzw. kein vollkommenes Modell. De Broglie beschrieb Elektronen und andere Teilchen als Welle. Er beschrieb dies mit der Energie E eines Photons, welche über die Planck-Gleichung mit der Lichtfrequenz in einer Beziehung steht.

\section{Orbitalanzahl pro Schale}
Schalen 1-4 mit max. Orbitalanzahl
\begin{center}
\begin{tabular}{l|c|c|r}
Schalen & s-Orbitale & p-Orbitale & d-Orbitale\\
\hline
1. Schale & 1 s-Orbital & keine & keine \\
2. Schale & 1 s-Orbital & 2 p-Orbitale & keine\\
3. Schale & 1 s-Orbital & 2 p-Orbitale & 2 d-Orbital\\
\end{tabular}
\end{center}
\section{Schroedinger-Gleichung}
\begin{center}
$-\frac{h^{2}}{2m}\frac{\delta^{2}\psi(x,t)}{\delta(x^{2})}+V\psi=E\psi(x)$
\end{center}
\end{document}
